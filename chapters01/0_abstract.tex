The current Internet is vulnerable to various attacks, e.g., source spoofing and flow hijacking attacks, which can be raised by misconfigurations or compromising routers. Unfortunately, both users and network operators are unable to localize these faults. Existing fault localization mechanisms detect such attacks under an assumption that localization is performed upon reliable communication channels. Unfortunately, detection channels of localization are not always reliable. In particular, adversaries can interfere with the fault localization. In this paper, we will relax the assumption and propose a robust data-plane fault localization protocol, called \name{},  
%\textcolor{red}{source and path verification protocol} 
that aims to localize faults and achieve source authenticity and path compliance in unreliable communication channels. \name{} allows different network entities to verify and sample packets so that packet source can efficiently localize faults of packet forwarding by verifying the sampling results. %In particular, 
Fault localization builds upon packet sampling, which is not impacted by the reliability of the communication channels. 
In particular, \name{} leverages a symmetric key distribution scheme to implement robust key distribution among different entities, which ensures that packet sources can always correctly fresh their keys to perform correct localization. 
%For enhancing reliable packet delivery, we present a robust fault localization scheme, named \name{}, to localize the misbehaved entity.} %\name{} uses symmetric keys to build secure detection channels and 
%\textcolor{red}{Based on acknowledgements from entities, \name{} and \namekey{} can enable the source to localize the fault during symmetric key distribution and packet delivery, which are} %enables packets sampling for localization on the channels such that it can detect and localize faults. In particular, the localization performed is 
%not impacted by the reliability of the communication channels, e.g., the packets used to localize faults are dropped. 
Our security and theoretical analysis prove the robustness of RFL protocol. We implement the \name{} prototype on the Click routers. The experimental results with the prototype demonstrate that \name{} achieves more than 99.5\% localization accuracy while incurring only 10\% throughput degradation.
%The current network is vulnerable to the source spoofing and flow hijacking attacks caused by compromised/misconfigured router. Both users and network operators are all powerless to localize the fault when any error occurs. Existing fault localization mechanisms cannot achieve a practical tradeoff between security and robustness, and require a unacceptably reliable transmission channel. In this paper, we propose a robust and lightweight data-plane fault localization protocol called RLFL for source authenticity and path compliance. RLFL provides anti-attack symmetric key distribution and fault localization in unreliable transmission channel, and lightweight router storage overhead of only 3.23 MB. We implemented prototype of RLFL on Linux/Click router, which achieves over 90\% throughput and approximate 85\% goodput of baseline. Our simulation results shows a higher localization accuracy of over 99.5\% while retaining a high level of security.

%\boldmath
%The current Internet is vulnerable to various types of the inconsistency between data plane and control plane, due to either malicious attacks, such as source spoofing and flow hijacking, or network misconfiguration, such as operator's incorrect manipulation. Unfortunately, some simple methods like ping, traceroute, etc. could not satisfy the requirement to monitor and localize this inconsistency as the possible existence of malicious intermediate router(s). Consequently, there is no reliable assurance provided for the current network to monitor this inconsistency, allowing counterfeited packet origins, eavesdropping private information and increased transmission overhead. Thus, detecting the data-plane source authenticity and path compliance is high desired to identify, locate the misbehaver 