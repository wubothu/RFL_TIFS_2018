\noindent{\textbf{Secure routing and forwarding.}}
Routing security has been widely studied to ensure correct packet forwarding on the Internet~\cite{kent2000secure} \cite{hu2004spv} \cite{goodell2003working} \cite{ng2004extensions} \cite{candolin2005packet} \cite{hu2013general}. S-BGP \cite{kent2000secure} verified the authenticity of announced routing paths by signing them, which incurs significant computation and communication overhead. In order to reduce the costs, a large amount of variants have been proposed. For example, So-BGP \cite{ng2004extensions} ensured correctness of announced routing paths by leveraging network topologies. IRV~\cite{goodell2003working} validated the correctness of the announced routing paths by establishing an additional IRV server in each AS, which limited its deploymentability. All these approaches did not address the security of routing data plane.\\
%which brings about the difficulties and challenges of maintenance and migration.
\noindent{\textbf{Source and path verification.}}
%There are many researches about source authentication and path validation.
%A scalable and secure OPT (
Origin and Path Trace (OPT) protocol \cite{kim2014lightweight} \cite{zhang2014mechanized} allows each router to verify delivered packets so as to verify correctness of packet source and forwarding paths. It reduced storage overhead in routers, which prevents state exhaustion attack. Naous et. al., \cite{naous2011verifying} proposed a Path Verification Mechanism (PVM) %to achieve the source and path verification,
to validate whether the packets correctly forwarded their forwarding paths. Cai et. al.,~\cite{cai2015source} performed source authentication and path validation by leveraging a set of orthogonal sequences instead of lightweight cryptographic operations. Unfortunately, these mechanism cannot localize the detected faults.
%Moreover, they might be v if they run in unreliable networks ({\bf LQ: double check, and tell more specific points.}) and . %the fault localization service, ignoring the investigation of the error. Other methods, such as
Although Passport \cite{liu2008passport} and SNAPP \cite{barak2008protocols} did not have such a  problem, they were vulnerable to source spoofing or path deviation attacks.\\
%cannot provide both packet source and forwarding path  or verify one aspect of source authenticity and path consistency.
\noindent{\textbf{Fault localization.}}
%To locate the misbehaved router or link, Adrian Perrig \emph{et al.} have carried out much research to
There are a large amount of studies on locating data plane errors~\cite{basescu2016high}~\cite{zhang2012shortmac}~\cite{zhang2012secure}~\cite{zhang2011network} \cite{wang2017smartfix}. Faultprints \cite{basescu2016high} was the first secure inter-domain fault localization scheme. It could localize the misbehaved links that drop, delay, modify packets at a high speed. ShortMAC \cite{zhang2012shortmac} leveraged probabilistic packet authentication to locate the illegal network links, which achieved low detection delay and incurred small overhead. DynaFL \cite{zhang2012secure} proposed the secure neighborhood-based fault localization (FL) protocol to  cope with dynamic traffic patterns and routing path with small router state. TrueNet \cite{zhang2011network} leveraged trusted computing technology to build a trusted network-layer architecture, and implemented a small TCB to address secure FL with small router state. However, these schemes might failed to localize faults if the generated acknowledgement packets are maliciously dropped by colluding entities. Our proposed protocol well addresses this issue by setting the timer on each entity, where the entity would send its sampling information towards {\tt S} once the timer expires. 